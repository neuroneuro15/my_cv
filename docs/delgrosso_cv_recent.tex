\documentclass{article}%
\usepackage[T1]{fontenc}%
\usepackage[utf8]{inputenc}%
\usepackage{lmodern}%
\usepackage{textcomp}%
\usepackage{lastpage}%
\usepackage{marginnote}%
\reversemarginpar%
\usepackage{graphicx}%
\usepackage[nochapters]{classicthesis}%
\usepackage[LabelsAligned]{currvita}%
\usepackage{hyperref}%
\hypersetup{colorlinks, breaklinks, urlcolor=Maroon, linkcolor=Maroon}%
\newlength{\datebox}%
\settowidth{\datebox}{Tuebingen, Germany}%
\renewcommand{\cvheadingfont}{\LARGE}%
\newcommand{\SubHeading}[1]{\vspace{1em}\noindent\spacedlowsmallcaps{#1}\vspace{0.7em}\\}%
\newcommand{\Email}[1]{\href{mailto:#1}{#1}}%
\newcommand{\MarginText}[1]{\marginpar{\raggedleft\small#1}}%
\newcommand{\Description}[1]{\hangindent=2em\hangafter=0\footnotesize{#1}\par\normalsize\vspace{1em}}%
\newcommand{\DescMarg}[2]{\noindent\hangindent=2em\hangafter=0 \parbox{\datebox}{\small} \MarginText{#1} #2 \vspace{0.3em}\\}%
\newcommand{\HeaderOnly}[2]{\noindent\hangindent=2em\hangafter=0 \parbox{\datebox}{\small \textit{#1}}\hspace{1.5em} #2 \vspace{0.5em}\\}%
\newcommand{\EntryHeader}[3]{\noindent\hangindent=2em\hangafter=0 \parbox{\datebox}{\small \textit{#2}}\hspace{1.5em} \MarginText{#1} #3 \vspace{0.5em}}%
\newcommand{\NewEntry}[4]{\EntryHeader{#1}{#2}{#3}\\\Description{#4}}%
%
%
%
\begin{document}%
\normalsize%
\thispagestyle{empty}%
\raggedright%
\begin{cv}{\spacedallcaps{Nicholas A. Del Grosso}}%
\vspace{2em}%
\SubHeading{Personal Info}%
\HeaderOnly{Address}{Karl{-}Witthalm{-}Str. 3, 81375 Muenchen}%
\HeaderOnly{Telephone}{+49 170 8253289}%
\HeaderOnly{E{-}mail}{\Email{delgrosso.nick@gmail.com}}%
\SubHeading{Goals}%
\begin{itemize}%
\item%
Support open science by building tools and teaching research methodology that promotes reproducible research.%
\end{itemize}%
\begin{itemize}%
\item%
Build technical skills in a wide variety of fields in order to perform high{-}quality research at institutes with limited resources.%
\end{itemize}%
\begin{itemize}%
\item%
Build explanatory models for sensorimotor learning that further our understanding of motor planning and cognition in biological systems.%
\end{itemize}%
\begin{itemize}%
\item%
Obtain teaching, project management, and laboratory experience sufficient to one day become an excellent university professor.%
\end{itemize}%
\SubHeading{Education}%
\NewEntry{Oct 2014 {-} Dec 2018}{PhD. Candidate}{Graduate School of Systemic Neurosciences, Luedwig{-}Maximillians Universitaet}{}%
\NewEntry{Aug 2012}{M.Sc. Neuroscience}{Graduate Training Centre of Neuroscience, Eberhard Karls Universitaet Tuebingen}{}%
\NewEntry{May 2010}{B.Sc. Psychology}{Wittenberg University}{}%
\SubHeading{Research Experience}%
\NewEntry{Dec 2019 {-}\newline%
  March 2020}{Ludwig{-}Maximillians Universitaet}{Dr. Thomas Wachtler}{Designed a short course on "Research Data Management" as part of the NFDI initiative on computational infrastructure training for neuroscience.}%
\NewEntry{Nov 2018 {-}\newline%
  July 2019}{Max Planck Institute of Biochemistry}{Prof. Dr. Matthias Mann}{Programmed high{-}throughput automated data collection and data analysis pipelines. I also designed and implemented a job{-}scheduling web application, implementing lean management methods to decrease data collection waiting times for 40 users and trained and mentored several biology and bioinformatics researchers in Python programming methods and open{-}source collaboration workflows, as well as gave introductory programming workshops for over 150 researchers.}%
\NewEntry{May 2013 {-}\newline%
  Nov 2018}{Ludwig{-}Maximillians Universitaet}{Prof. Dr. Anton Sirota}{Programmed a 3D graphics engine in Python to build virtual reality system for freely moving rats, supervised students in programming, engineering, and cognitive science projects, organized weekly journal clubs, and ordered new equipment, trained rodents to perform behavioral tasks, and performed surgery on said rodents as part of brain research.}%
\SubHeading{Industry Experience}%
\NewEntry{}{Freelance Data Analysis Programming Trainer}{.}{I teach week{-}long programming workshops for research institutes, universities, and private companies.}%
\SubHeading{Journal Publications}%
\Description{Nicholas A. Del Grosso, Anton Sirota.  ``Ratcave, A 3D graphics python package for cognitive psychology experiments'' May 2019. Behavioral Research Methods. https://doi.org/10.3758/s13428{-}019{-}01245{-}x}%
\Description{Nicholas A. Del Grosso, Justin J. Graboski, Weiwei Chen, Eduardo Blanco Hernández, Anton Sirota. ``Virtual Reality system for freely{-}moving rodents.'' bioRxiv 161232. July 2017; doi=https://doi.org/10.1101/161232}%
\Description{Broetz D., Del Grosso, N.A., Rea M., Ramos{-}Murguialday, A., Soekadar S.R., Birbaumer, N. ``A New Hand Assessment Instrument for Severely Affected Stroke Patients.''  Journal of Neurorehabilitation. 2014; 34(3), 409{-}27.}%
\Description{Benoit, J.B., Del Grosso, N.A., Yoder, J.A., Denlinger, D.L. ``Resistance to Dehydration between Bouts of Blood Feeding in the Bed Bug, Cimex Lectularius, is Enhanced by Water Conservation, Aggregation, and Quiescence.'' American Journal of Tropical Medical Hygience. May 2007; 76(5), 987{-}93.}%
\SubHeading{Conference Publications}%
\NewEntry{September 2018}{Harvard{-}LMU Young Scientists Forum}{Testing CAVE virtual reality systems for use in animal behavior research}{}%
\NewEntry{November 2017}{Society for Neuroscience}{Generalized Rat Spontaneous Behavior in a CAVE Experimental Setup.}{}%
\NewEntry{July 2017}{PyData Barcelona}{The Neuroscience Lab; A Tour Through the Eyes of a Pythonista}{}%
\NewEntry{November 2016}{Munich Interact}{Tracking Rats Exploring a Virtual World; Do They Believe what they See?}{}%
\NewEntry{July 2016}{FENS Forum of Neuroscience}{Probing Rodent Perception of Virtual Environments with Freely{-}Moving Virtual Reality}{}%
\NewEntry{June 2015}{Synergy Munich}{ratCAVE, A Novel Virtual Reality System for Freely{-}Moving Rodents}{}%
\NewEntry{March 2015}{Interact Munich}{Demonstrating a Freely{-}Moving Virtual Reality Approach for Rodent Research}{}%
\NewEntry{Nov 2014}{Society for Neuroscience}{ratCAVE, A Novel Virtual Reality System for Freely{-}Moving Rodents.}{}%
\NewEntry{Nov. 2012}{NENA Tuebingen}{Interpreting (M)EEG, A First Look at Dynamic Causal Modeling.}{Introduced a probabilistic nonlinear modeling framework for interpretation of MEG and EEG data, along with the results of a pilot study in which we applied the approach.}%
\NewEntry{Nov. 2011}{NENA Tuebingen}{The Intrinsic Bias During the Blind{-}Walking Task is Not Caused by an Aberrant Intrinsic Ground{-}Slope Model.}{}%
\SubHeading{Skills}%
\begin{itemize}%
\item%
\textbf{Languages}: English (Mother Tongue), German (Level B1), French (Level A1-2)%
\end{itemize}%
\begin{itemize}%
\item%
\textbf{Programming}: Python, R, Matlab, GLSL, Docker, LabView, C, Bash/Linux, LaTeX, Arduino%
\end{itemize}%
\begin{itemize}%
\item%
\textbf{Stimulus Presentation}: Psychopy, Neurobs Presentation, Psychophysics Toolbox, OpenGL, Pyglet, Vispy, SuperLab, RatCAVE%
\end{itemize}%
\begin{itemize}%
\item%
\textbf{Statistical Analysis}: Python SciPy Stack (Pandas, Numpy, Matplotlib, etc), Statistical Parametric Mapping (SPM), SPSS, R, Matlab Statistics Toolbox, Fieldtrip, gTec Analyze, BrainVision Analyzer%
\end{itemize}%
\begin{itemize}%
\item%
\textbf{Data Workflow Management}: Snakemake, PyDoit, Docker, Singularity%
\end{itemize}%
\begin{itemize}%
\item%
\textbf{Graphics}: Blender3D, Adobe Suite (Photoshop, Illustrator, and InDesign), OpenGL, Google SketchUp, Open Source Suite (GIMP, Inkspace, and Scribus)%
\end{itemize}%
\begin{itemize}%
\item%
\textbf{Wet Lab Skills}: Rat Neurosurgery, Animal Behavioral training (rats and monkeys), in vivo electrophysiology (single needle electrodes, chronically-implanted electrode arrays, noninvasive arrays of EEG electrodes and MEG sensors), Basic Electronics, Comfortable with building custom laboratory equipment%
\end{itemize}%
\begin{itemize}%
\item%
\textbf{EEG System Experience}: BrainProducts, gTec, Grass Instruments, CTF%
\end{itemize}%
Full List of Positions and Publications Available Upon Request.%
\end{cv}%
\end{document}