\documentclass{article}%
\usepackage[T1]{fontenc}%
\usepackage[utf8]{inputenc}%
\usepackage{lmodern}%
\usepackage{textcomp}%
\usepackage{lastpage}%
\usepackage{marginnote}%
\reversemarginpar%
\usepackage{graphicx}%
\usepackage[nochapters]{classicthesis}%
\usepackage[LabelsAligned]{currvita}%
\usepackage{hyperref}%
\hypersetup{colorlinks, breaklinks, urlcolor=Maroon, linkcolor=Maroon}%
\newlength{\datebox}%
\settowidth{\datebox}{Tuebingen, Germany}%
\renewcommand{\cvheadingfont}{\LARGE}%
\newcommand{\SubHeading}[1]{\vspace{1em}\noindent\spacedlowsmallcaps{#1}\vspace{0.7em}\\}%
\newcommand{\Email}[1]{\href{mailto:#1}{#1}}%
\newcommand{\GitHub}[1]{\href{https://www.github.com/#1}{#1}}%
\newcommand{\MarginText}[1]{\marginpar{\raggedleft\small#1}}%
\newcommand{\Description}[1]{\hangindent=2em\hangafter=0\footnotesize{#1}\par\normalsize\vspace{1em}}%
\newcommand{\DescMarg}[2]{\noindent\hangindent=2em\hangafter=0 \parbox{\datebox}{\small} \MarginText{#1} #2 \vspace{0.3em}\\}%
\newcommand{\HeaderOnly}[2]{\noindent\hangindent=2em\hangafter=0 \parbox{\datebox}{\small \textit{#1}}\hspace{1.5em} #2 \vspace{0.5em}\\}%
\newcommand{\EntryHeader}[3]{\noindent\hangindent=2em\hangafter=0 \parbox{\datebox}{\small \textit{#2}}\hspace{1.5em} \MarginText{#1} #3 \vspace{0.5em}}%
\newcommand{\NewEntry}[4]{\EntryHeader{#1}{#2}{#3}\\\Description{#4}}%
%
%
%
\begin{document}%
\normalsize%
\thispagestyle{empty}%
\raggedright%
\begin{cv}{\spacedallcaps{Nicholas A. Del Grosso}}%
\vspace{2em}%
\SubHeading{Personal Info}%
\HeaderOnly{Address}{Karl{-}Witthalm{-}Str. 3, 81375 Muenchen}%
\HeaderOnly{Telephone}{+49 170 8253289}%
\HeaderOnly{E{-}mail}{\Email{delgrosso.nick@gmail.com}}%
\HeaderOnly{GitHub}{\GitHub{nickdelgrosso}}%
\SubHeading{Goals}%
\begin{itemize}%
\item%
Support open science by building tools and teaching research methodology that promotes reproducible research.%
\end{itemize}%
\begin{itemize}%
\item%
Build technical skills in a wide variety of fields in order to perform high{-}quality research at institutes with limited resources.%
\end{itemize}%
\begin{itemize}%
\item%
Build explanatory models for sensorimotor learning that further our understanding of motor planning and cognition in biological systems.%
\end{itemize}%
\begin{itemize}%
\item%
Obtain teaching, project management, and laboratory experience sufficient to one day become an excellent university professor.%
\end{itemize}%
\SubHeading{Education}%
\NewEntry{Oct 2014 {-} Dec 2018}{PhD. Candidate}{Graduate School of Systemic Neurosciences, Luedwig{-}Maximillians Universitaet}{}%
\NewEntry{Aug 2012}{M.Sc. Neuroscience}{Graduate Training Centre of Neuroscience, Eberhard Karls Universitaet Tuebingen}{}%
\NewEntry{May 2010}{B.Sc. Psychology}{Wittenberg University}{}%
\SubHeading{Research Experience}%
\NewEntry{Dec 2019 {-}\newline%
  March 2020}{Ludwig{-}Maximillians Universitaet}{Dr. Thomas Wachtler}{Designed a short course on "Research Data Management" as part of the NFDI initiative on computational infrastructure training for neuroscience.}%
\NewEntry{Nov 2018 {-}\newline%
  July 2019}{Max Planck Institute of Biochemistry}{Prof. Dr. Matthias Mann}{Programmed high{-}throughput automated data collection and data analysis pipelines. I also designed and implemented a job{-}scheduling web application, implementing lean management methods to decrease data collection waiting times for 40 users and trained and mentored several biology and bioinformatics researchers in Python programming methods and open{-}source collaboration workflows, as well as gave introductory programming workshops for over 150 researchers.}%
\NewEntry{May 2013 {-}\newline%
  Nov 2018}{Ludwig{-}Maximillians Universitaet}{Prof. Dr. Anton Sirota}{Programmed a 3D graphics engine in Python to build virtual reality system for freely moving rats, supervised students in programming, engineering, and cognitive science projects, organized weekly journal clubs, and ordered new equipment, trained rodents to perform behavioral tasks, and performed surgery on said rodents as part of brain research.}%
\NewEntry{Aug 2012 {-}\newline%
  May 2013}{Universitaet Tuebingen}{Prof. Dr. Christoph Braun}{Wrote a research grant to study the top{-}down and bottom{-}up interactions by computational modeling information propagation in early sensory pathways as measured by MEG, designed and administrated an institute wiki, organized a student lecture series, and supervised two students’ EEG research projects.}%
\NewEntry{Nov 2011 {-}\newline%
  July 2012}{Universitaet Tuebingen}{Prof. Dr. Niels Birbaumer}{Programmed in Matlab a time{-}frequency and evoked potential analysis on three years’ worth of MEG data assessing longitudinal changes in stroke patients receiving physiotherapy.}%
\NewEntry{Oct 2012 {-}\newline%
  Nov 2012}{Universitaet Tuebingen}{Prof. Dr. Cornelius Schwarz}{Trained rats to perform whisking in response to barrel cortex stimulation viachronically{-}implanted electrodes, mapping stimulation sensitivity to each cortical layer.}%
\NewEntry{Nov 2010 {-}\newline%
  March 2011}{Universitaet Tuebingen}{Dr. Michael Barnett Cowan}{Programmed an online EMG classifier in Matlab and Simulink to accurately detect finger movements within milliseconds for EEG coherence brain{-}computer interface training.}%
\NewEntry{Dec. 2009 {-}\newline%
  Aug. 2010}{Wittenberg University}{Prof. Dr. Josephine Wilson}{Built an NI{-}DAQ EEG system, programmed online analysis and data acquisition in Matlab and LabView, and confirmed its functionality in three different experiments. As a senior lab assistant, also worked as an aid for rat neurosurgery and noninvasive electrophysiology (skin conductance, EMG, EKG, and EEG) laboratory course sessions, which included planning and giving demonstrations on each method above.}%
\NewEntry{June{-}Aug 2008 {-}\newline%
  June 2009}{Duke University}{Prof. Dr. Jennifer Groh}{Trained Macaque monkeys to perform visual saccade tasks while mapping receptive fields in superior and inferior colliculus.}%
\NewEntry{Aug 2007 {-}\newline%
  Dec. 2009}{Wittenberg University}{Prof. Dr. Michael Anes}{Conducted three behavioral psychophysics studies on the hemispheric lateralization of face perception. Tasks included programming stimulus sequences in SuperLab, patient recruitment and management, data collection, and conference poster preparation.}%
\NewEntry{Nov 2006 {-}\newline%
  March 2007}{Wittenberg University}{Prof. Dr. Jay Yoder}{Measured dessication rates in the bed bug and isolated fungal growth in three species of cockroach. These studies resulted in a publication in a peer{-}reviewed journal and a poster presentation at an undergraduate research conference.}%
\SubHeading{Industry Experience}%
\NewEntry{}{Freelance Data Analysis Programming Trainer}{.}{I teach week{-}long programming workshops for research institutes, universities, and private companies.}%
\NewEntry{}{Freelance Scientific Consultant}{UKT Psychosomatic Med. and Sports Med.}{I evaluated and designed a solution for performing medical science studies in a placebo study, and taught the PhD student who carried out the study over several remote sessions and a few travel consultations.}%
\NewEntry{}{Research Internship}{The Neuromarketing Labs}{I completed set{-}up of an EEG laboratory, including software calibration and noise measurements. Designed and ran two experiments estimating the evoked responses of semantic agreement and price agreement, then analyzed the data. The results from the second experiment are the basis of Dr. Mueller’s recently{-}published book, "Neuropricing". Currently volunteering as an EEG consultant by giving one{-}day workshops on Fieldtrip, SPM, and artifact correction methods.}%
\SubHeading{Teaching Experience}%
\NewEntry{April 2019}{Trainer}{The Python Tsunami}{In this compressed version of the "Intro. to Python" course, I led 100 biology students and 6 instructors in a conference{-}style 2{-}day course that culminated in a data jam showcasing their new programming and data analysis skills.}%
\NewEntry{March 2019}{Trainer}{Computer Graphics for Virtual Reality}{Taught Object{-}Oriented Programming methdology in the context of designing a 3D Graphics library, including 3D matrix multiplication, shader graphics, and event loop programming.}%
\NewEntry{August 2018}{Trainer}{Int. Scientific Programming in Python}{A 4{-}day workshop on good programming practices essential for writing robust, reusable Python code for scientific data analysis.  Covered object{-}oriented programming, advanced iteration and automation structures in Python, performance profiling, testing, and packaging code.  Throughout each unit, I emphasized idiomatic code style, good coding practices, and problem{-}solving strategies.}%
\NewEntry{July 2018}{Programming Instructor}{Teaching Statistics with R}{This 4{-}day workshop is an intensive R course taught to the Psychology department professors at Kwantlen University in Vancouver, Canada.  In this course, they learned the R programming language and how to build statistics teaching materials with it.}%
\NewEntry{April 2018 {-} Present}{Organizer}{Munich Science Slam}{Co{-}founded a tri{-}monthly science slam event series, working with 10{-}14 speakers from various institutes to give talks in a competion format.  At the end of each event, I demonstrate a real{-}time evaluation system into the event for the 60 audience members to use for feedback.}%
\NewEntry{April 2018 {-} Present}{Soft Skills Trainer}{Presentation Skills for Scientists}{With this 1{-}2 day workshop, I taught effective talk organization and outlining skills for scientific presentations.  Scientists gain confidence by learning how to focus on their goals and delivering speeches convincingly through repeated practice.}%
\NewEntry{October 2017}{3D Graphics Instructor}{Animal Tracking and VR Bootcamp}{I co{-}taught an international, week{-}long workshop on combining animal tracking through machine vision methods and 3D graphics applications to build virtual reality systems for freely{-}moving animals.  Besides theoretical lectures on the mathematics and engineering behind virtual reality systems, I wrote tutorials for software I wrote to teach the concepts, from which the participants, consisting of PhD students, Post{-}docs, and Professors, built their own prototype VR systems for ants.}%
\NewEntry{Fall 2017 {-} Present}{Organizer}{PyData Munich}{I revived a local chapter for the global PyData organization, coordinating with technology companies in Munich (e.g. Google, Nokia, TNG Consulting, JetBrains, and Wayra) to build a data{-}science teaching community through the MeetUp platform. These companies now host biweekly tutorials at their event spaces, sponsoring each event and providing spaces for university researchers and tech industry specialists to meet, interact, and learn together.}%
\NewEntry{Summer 2017}{Organizer}{Super Python Talks for Life Science}{I organized a biweekly seminar series for teaching intermediate{-}level data analysis and Python programming tutorials, given by 10 PhD students and Pos{-}docs, including myself.  Besides recruiting these speakers, I organized the room and equipment for these sessions, advertised the events, and ran the sessions.  This series was successful; it was regularly attended by 30{-}70 researchers.}%
\NewEntry{July 2016 {-} Present}{Trainer}{Intro. to Scientific Programming in Python}{This 4{-}day workshop is an intensive version of the semester Python course I teach at LMU every few months.  In this period, students with no programming experience gain the skills needed to perform data analysis and in Python and reason about their analysis workflow.}%
\NewEntry{Summer 2016 {-} Summer 2017}{Lecturer}{Intro. to Scientific Programming in Python}{In this semester course, taught two years in a row, I taught beginning programmers data management, scientific data analysis, and programming skills in a new language (Python).  Besides organizing and planning the course, I also prepared all course materials, homework assignments, and graded their final projects.}%
\NewEntry{Winter 2013 {-} Summer 2014}{Lecturer}{Introduction to Matlab}{For 3 Semesters, I taught a 2{-}week introduction to programming course to beginning programmers.  Besides organizing, planning, and teaching the course, I also prepared all course materials and homework assignments.}%
\NewEntry{December 2015}{Teaching Assistant}{Psychophysics}{In this 2{-}week block course, I acted as tutor, providing technical and programming assistance to students programming and analysing their own psychopysics experiments in Matlab, R, and Excel.}%
\NewEntry{2015 {-} 2017}{Proofreader}{Freelance Proofreader}{Proofread and Edited research papers for graduate students in medicine, neuroscience, and philosophy to programming}%
\SubHeading{Journal Publications}%
\Description{Nicholas A. Del Grosso, Anton Sirota.  ``Ratcave, A 3D graphics python package for cognitive psychology experiments'' May 2019. Behavioral Research Methods. https://doi.org/10.3758/s13428{-}019{-}01245{-}x}%
\Description{Nicholas A. Del Grosso, Justin J. Graboski, Weiwei Chen, Eduardo Blanco Hernández, Anton Sirota. ``Virtual Reality system for freely{-}moving rodents.'' bioRxiv 161232. July 2017; doi=https://doi.org/10.1101/161232}%
\Description{Broetz D., Del Grosso, N.A., Rea M., Ramos{-}Murguialday, A., Soekadar S.R., Birbaumer, N. ``A New Hand Assessment Instrument for Severely Affected Stroke Patients.''  Journal of Neurorehabilitation. 2014; 34(3), 409{-}27.}%
\Description{Benoit, J.B., Del Grosso, N.A., Yoder, J.A., Denlinger, D.L. ``Resistance to Dehydration between Bouts of Blood Feeding in the Bed Bug, Cimex Lectularius, is Enhanced by Water Conservation, Aggregation, and Quiescence.'' American Journal of Tropical Medical Hygience. May 2007; 76(5), 987{-}93.}%
\SubHeading{Conference Publications}%
\NewEntry{September 2018}{Harvard{-}LMU Young Scientists Forum}{Testing CAVE virtual reality systems for use in animal behavior research}{}%
\NewEntry{November 2017}{Society for Neuroscience}{Generalized Rat Spontaneous Behavior in a CAVE Experimental Setup.}{}%
\NewEntry{July 2017}{PyData Barcelona}{The Neuroscience Lab; A Tour Through the Eyes of a Pythonista}{}%
\NewEntry{November 2016}{Munich Interact}{Tracking Rats Exploring a Virtual World; Do They Believe what they See?}{}%
\NewEntry{July 2016}{FENS Forum of Neuroscience}{Probing Rodent Perception of Virtual Environments with Freely{-}Moving Virtual Reality}{}%
\NewEntry{June 2015}{Synergy Munich}{ratCAVE, A Novel Virtual Reality System for Freely{-}Moving Rodents}{}%
\NewEntry{March 2015}{Interact Munich}{Demonstrating a Freely{-}Moving Virtual Reality Approach for Rodent Research}{}%
\NewEntry{Nov 2014}{Society for Neuroscience}{ratCAVE, A Novel Virtual Reality System for Freely{-}Moving Rodents.}{}%
\NewEntry{Nov. 2012}{NENA Tuebingen}{Interpreting (M)EEG, A First Look at Dynamic Causal Modeling.}{Introduced a probabilistic nonlinear modeling framework for interpretation of MEG and EEG data, along with the results of a pilot study in which we applied the approach.}%
\NewEntry{Nov. 2011}{NENA Tuebingen}{The Intrinsic Bias During the Blind{-}Walking Task is Not Caused by an Aberrant Intrinsic Ground{-}Slope Model.}{}%
\NewEntry{April 2010}{Visual Sciences Society}{DIY ERPs, Designing inexpensive EEG systems for performing auditroy and visual cognitive studies.}{}%
\NewEntry{March 2010}{Butler Undergraduate Research Conference}{Discrimination and processing of deviant stimuli at the auditory cortex.}{}%
\NewEntry{Sep. 2009}{European Health Psychology Society}{Discrimination of attention{-}related and motor{-}related evoked activity by hemispheric comparison over the motor cortex.}{}%
\NewEntry{May 2009}{Visual Sciences Society}{Are Local Changes in Faces Really Local?}{}%
\NewEntry{May 2008}{Visual Sciences Society}{Hemispheric specialization for face processing revealed by use of thatcherized and feature{-}distorted faces.}{}%
\SubHeading{Skills}%
\begin{itemize}%
\item%
\textbf{Languages}: English (Mother Tongue), German (Level B1), French (Level A1-2)%
\end{itemize}%
\begin{itemize}%
\item%
\textbf{Programming}: Python, R, Matlab, GLSL, Docker, LabView, C, Bash/Linux, LaTeX, Arduino%
\end{itemize}%
\begin{itemize}%
\item%
\textbf{Stimulus Presentation}: Psychopy, Neurobs Presentation, Psychophysics Toolbox, OpenGL, Pyglet, Vispy, SuperLab, RatCAVE%
\end{itemize}%
\begin{itemize}%
\item%
\textbf{Statistical Analysis}: Python SciPy Stack (Pandas, Numpy, Matplotlib, etc), Statistical Parametric Mapping (SPM), SPSS, R, Matlab Statistics Toolbox, Fieldtrip, gTec Analyze, BrainVision Analyzer%
\end{itemize}%
\begin{itemize}%
\item%
\textbf{Data Workflow Management}: Snakemake, PyDoit, Docker, Singularity%
\end{itemize}%
\begin{itemize}%
\item%
\textbf{Graphics}: Blender3D, Adobe Suite (Photoshop, Illustrator, and InDesign), OpenGL, Google SketchUp, Open Source Suite (GIMP, Inkspace, and Scribus)%
\end{itemize}%
\begin{itemize}%
\item%
\textbf{Wet Lab Skills}: Rat Neurosurgery, Animal Behavioral training (rats and monkeys), in vivo electrophysiology (single needle electrodes, chronically-implanted electrode arrays, noninvasive arrays of EEG electrodes and MEG sensors), Basic Electronics, Comfortable with building custom laboratory equipment%
\end{itemize}%
\begin{itemize}%
\item%
\textbf{EEG System Experience}: BrainProducts, gTec, Grass Instruments, CTF%
\end{itemize}%
\SubHeading{Awards}%
\DescMarg{October 2017}{Hackathon 3rd Place Winner and "Most Creative Team" Award at Burda Bootcamp Event "Health and Fitness Hackathon"}%
\DescMarg{July 2017}{Hackathon Track Winner at Media Lab Bayern Event "FutureLab{-}{-}Smart Home meets Journalism"}%
\DescMarg{April 2017}{Hackathon Winner at Burda Bootcamp Event "Love Hackathon"}%
\DescMarg{2016}{Best Talk Award at Interact Munich Conference}%
\DescMarg{2015}{Best Poster Award at Interact Munich Conference}%
\DescMarg{2011}{National Science Foundation Graduate Research Fellowship}%
\DescMarg{2008}{NSF Neuroscience REU Fellowship at Duke University}%
\end{cv}%
\end{document}