\documentclass{article}%
\usepackage[T1]{fontenc}%
\usepackage[utf8]{inputenc}%
\usepackage{lmodern}%
\usepackage{textcomp}%
\usepackage{lastpage}%
\usepackage{marginnote}%
\reversemarginpar%
\usepackage{graphicx}%
\usepackage[nochapters]{classicthesis}%
\usepackage[LabelsAligned]{currvita}%
\usepackage{hyperref}%
\hypersetup{colorlinks, breaklinks, urlcolor=Maroon, linkcolor=Maroon}%
\newlength{\datebox}%
\settowidth{\datebox}{Tuebingen, Germany}%
\renewcommand{\cvheadingfont}{\LARGE\color{Maroon}}%
\newcommand{\SubHeading}[1]{\vspace{1em}\noindent\spacedlowsmallcaps{#1}\vspace{0.7em}\\}%
\newcommand{\Email}[1]{\href{mailto:#1}{#1}}%
\newcommand{\MarginText}[1]{\marginpar{\raggedleft\small#1}}%
\newcommand{\Description}[1]{\hangindent=2em\hangafter=0\footnotesize{#1}\par\normalsize\vspace{1em}}%
\newcommand{\DescMarg}[2]{\noindent\hangindent=2em\hangafter=0 \parbox{\datebox}{\small} \MarginText{#1} #2 \vspace{0.3em}\\}%
\newcommand{\HeaderOnly}[2]{\noindent\hangindent=2em\hangafter=0 \parbox{\datebox}{\small \textit{#1}}\hspace{1.5em} #2 \vspace{0.5em}\\}%
\newcommand{\EntryHeader}[3]{\noindent\hangindent=2em\hangafter=0 \parbox{\datebox}{\small \textit{#2}}\hspace{1.5em} \MarginText{#1} #3 \vspace{0.5em}}%
\newcommand{\NewEntry}[4]{\EntryHeader{#1}{#2}{#3}\\\Description{#4}}%
%
%
%
\begin{document}%
\normalsize%
\thispagestyle{empty}%
\raggedright%
\begin{cv}{\spacedallcaps{Nicholas A. Del Grosso}}%
\vspace{2em}%
\SubHeading{Personliche Daten}%
\HeaderOnly{Adresse}{Karl{-}Witthalm{-}Str. 3, 81375 München}%
\HeaderOnly{Telefone}{+49 170 8253289}%
\HeaderOnly{Email}{\Email{delgrosso.nick@gmail.com}}%
\SubHeading{Ziele}%
\begin{itemize}%
\item%
Andere dazu inspirieren, durch Mentoring, Unterricht und Führung großartige Dinge zu erreichen.%
\end{itemize}%
\begin{itemize}%
\item%
Technische Fähigkeiten in den verschiedensten Bereichen zu entwickeln, um qualitativ hochwertige Forschung an Instituten mit begrenzten Ressourcen durchzuführen.%
\end{itemize}%
\begin{itemize}%
\item%
Unterstützung offener Wissenschaft durch den Aufbau von Werkzeugen und das Unterrichten von Forschungsmethoden, die reproduzierbare Forschung fördern.%
\end{itemize}%
\begin{itemize}%
\item%
Erhalten Sie Lehr{-}, Projektmanagement{-} und Laborerfahrung, um eines Tages ein kompetenter Universitätsprofessor zu werden.%
\end{itemize}%
\SubHeading{Ausbildung}%
\NewEntry{Oct 2014 {-} Present}{PhD. Neurowissenschaft}{Graduate School of Systemic Neurosciences, Lüdwig{-}Maximillians Üniversität}{}%
\NewEntry{Aug 2012}{M.Sc. Neurowissenschaft}{Max Planck International Research School, Graduate School of Neural and Behavioural Sciences}{}%
\NewEntry{May 2010}{B.Sc. Psychologie}{Wittenberg University}{}%
\SubHeading{Forschungserfahrung}%
\NewEntry{May 2013 {-}\newline%
  Present}{Ludwig{-}Maximillians Universität}{Prof. Dr. Anton Sirota}{Programmierte eine 3D{-}Grafik{-}Engine in Python, um ein Virtual{-}Reality{-}System für frei bewegliche Ratten aufzubauen, konzipierte und realisierte kognitionswissenschaftliche Experimente und testete die Verallgemeinerbarkeit der Virtual{-}Reality{-}Forschung an ihren realen Gegenstücken; betreute sechs Studenten in Programmier{-}, Ingenieur{-} und kognitionswissenschaftlichen Projekten organisierte wöchentliche Journalclubs, organisierte geplante soziale Veranstaltungen und Retreats und bestellte neue Laborgeräte.}%
\NewEntry{Aug 2012 {-}\newline%
  Mai 2013}{Universität Tübingen}{Prof. Dr. Christoph Braun}{Verfasste ein Forschungsstipendium zur Untersuchung der Top{-}Down{-} und Bottom{-}Up{-}Interaktionen durch computergestützte Modellierung von Informationsfortpflanzung in frühen sensorischen Pfaden nach MEG, konzipierte und verwaltete ein Instituts{-}Wiki, organisierte eine studentische Vorlesungsreihe und betreute zwei EEG{-}Forschungsprojekte von Schülern.}%
\NewEntry{Nov 2011 {-}\newline%
  July 2012}{Universität Tübingen}{Prof. Dr. Niels Birbaumer}{Programmierte eine Zeit{-}Frequenz{-} und Evozierte{-}Potential{-}Analyse in Matlab über drei Jahre an MEG{-}Daten, um Längsänderungen bei Schlaganfallpatienten, die Physiotherapie erhalten, zu beurteilen.}%
\NewEntry{Oct 2012 {-}\newline%
  Nov 2012}{Universität Tübingen}{Prof. Dr. Cornelius Schwarz}{In dieser Labordrehung trainierte ich Ratten, um als Reaktion auf die Kortexrhythmusstimulation über chronisch implantierte Elektroden ein Quirlen durchzuführen, wobei die Stimulationsempfindlichkeit jeder kortikalen Schicht zugeordnet wurde.}%
\NewEntry{Nov 2010 {-}\newline%
  March 2011}{Universität Tübingen}{Dr. Michael Barnett Cowan}{Programmierte einen Online{-}EMG{-}Klassifizierer in Matlab und Simulink, um Fingerbewegungen innerhalb von Millisekunden für EEG{-}Kohärenz Brain{-}Computer{-}Interface{-}Training genau zu erkennen.}%
\NewEntry{Dec. 2009 {-}\newline%
  Aug. 2010}{Wittenberg University}{Prof. Dr. Josephine Wilson}{Ich baute ein NI{-}DAQ{-}EEG{-}System, programmierte Online{-}Analyse und Datenerfassung in Matlab und LabView und bestätigte seine Funktionalität in drei verschiedenen Experimenten. Als leitende Laborassistentin arbeitete sie auch als Assistentin für Rattenneurochirurgie und nichtinvasive Elektrophysiologie (Hautleitfähigkeit, EMG, EKG und EEG) in Laborkursen, zu denen auch die Planung und Demonstration der einzelnen Methoden gehörten.}%
\NewEntry{June{-}Aug 2008 {-}\newline%
  June 2009}{Duke University}{Prof. Dr. Jennifer Groh}{Unterrichtete Makaken{-}Affen, um visuelle Sakkadenaufgaben durchzuführen, während sie rezeptive Felder im oberen und unteren Colliculus kartierten.}%
\NewEntry{Aug 2007 {-}\newline%
  Dec. 2009}{Wittenberg University}{Prof. Dr. Michael Anes}{Durchführung von drei verhaltenspsychophysischen Studien zur hemisphärischen Lateralisierung der Gesichtswahrnehmung. Zu den Aufgaben gehörten die Programmierung von Stimulussequenzen im SuperLab, die Rekrutierung und Verwaltung von Patienten, die Datenerfassung und die Vorbereitung von Konferenzplakaten.}%
\NewEntry{Nov 2006 {-}\newline%
  March 2007}{Wittenberg University}{Prof. Dr. Jay Yoder}{Gemessene Entwässerungsraten bei der Bettwanze und isoliertem Pilzwachstum bei drei Arten von Küchenschaben. Diese Studien führten zu einer Publikation in einem Peer{-}Review{-}Journal und einer Posterpräsentation bei einer Undergraduate{-}Forschungskonferenz.}%
\SubHeading{Berufserfahrung}%
\NewEntry{}{Freiberuflich}{Wissenschaftlicher Berater und Trainer}{Trainiere ich Forscher in Programmierung, Experimentierdesign und wissenschaftlichen Schreibfähigkeiten, indem Sie sie dabei unterstützen, ihre eigenen Lösungen für Forschungsprobleme zu entwickeln und einwöchige Schulungen für ihre Forschungsinstitute durchzuführen.}%
\NewEntry{}{Forschungspraktikum}{The Neuromarketing Labs}{Ich habe die Einrichtung eines EEG{-}Labors abgeschlossen, einschließlich Software{-}Kalibrierung und Rauschmessungen. Entwarf und führte zwei Experimente durch, die die evozierten Reaktionen der semantischen Übereinstimmung und der Preisvereinbarung schätzten und analysierte dann die Daten. Die Ergebnisse des zweiten Experiments sind die Grundlage des kürzlich erschienenen Buches von Dr. Müller, Neuropricing. Derzeit ehrenamtliche Tätigkeit als EEG{-}Berater durch eintägige Workshops zu Fieldtrip, SPM und Artefaktkorrekturmethoden.}%
\SubHeading{Lehrerfahrung}%
\NewEntry{July 2018}{Programming Instructor}{Lehrstatistik mit R}{Dieser viertägige Workshop ist ein intensiver R{-}Kurs, der den Professoren der Psychologieabteilung der Kwantlen Universität in Vancouver, Kanada, erteilt wird. In diesem Kurs lernten sie die R{-}Programmiersprache und lernten, wie sie damit statistisches Unterrichtsmaterial erstellen können.}%
\NewEntry{April 2018}{Veranstalter}{Munich Science Slam}{Organisation eines Science{-}Slam Verananstaltung in einer lokalen Veranstaltungshalle mit 14 Referenten aus 5 Instituten, um Vorträge zu halten. Integriert ein Echtzeit{-}Bewertungssystem in die Veranstaltung für die 60 Zuschauer für Feedback verwenden. Diese Veranstaltung war erfolgreich und wurde im Oktober 2018 wiederholt}%
\NewEntry{April 2018 {-} Present}{Soft Skills Trainer}{Präsentationsfähigkeiten für Wissenschaftler}{Mit diesem 1{-}2{-}tägigen Workshop unterrichtete ich effektive Sprechorganisation und skizzierte Fähigkeiten für wissenschaftliche Präsentationen. Wissenschaftler gewinnen Vertrauen, indem sie lernen, sich auf ihre Ziele zu konzentrieren und durch wiederholtes Üben überzeugend zu sprechen.}%
\NewEntry{October 2017}{3D Graphics Instructor}{Tierverfolgung und VR Bootcamp}{Ich habe einen internationalen, einwöchigen Workshop über die Kombination von Tierverfolgung mit Methoden der Bildverarbeitung und 3D{-}Grafikanwendungen gelehrt, um virtuelle Realitätssysteme für frei bewegliche Tiere zu bauen. Neben theoretischen Vorlesungen über Mathematik und Technik hinter Virtual Reality Systemen schrieb ich Tutorials für Software, die ich geschrieben habe, um die Konzepte zu unterrichten, aus denen die Teilnehmer, bestehend aus Doktoranden, Postdocs und Professoren, ihre eigenen Prototyp{-}VR{-}Systeme für Ameisen entwickelten.}%
\NewEntry{Fall 2017}{Veranstalter}{PyData Munich}{Ich habe ein lokales Kapitel für die globale PyData{-}Organisation wiederbelebt und mich mit Technologieunternehmen in München (z. B. Google, Nokia, TNG Consulting, JetBrains und Wayra) zusammengetan, um mithilfe der MeetUp{-}Plattform eine datenwissenschaftliche Lehrgemeinschaft aufzubauen. Diese Unternehmen veranstalten jetzt zweiwöchentliche Tutorien in ihren Veranstaltungsräumen, sponsern jede Veranstaltung und bieten Raum für Universitätsforscher und Experten der Technologiebranche, um sich zu treffen, zu interagieren und gemeinsam zu lernen.}%
\NewEntry{Summer 2017}{Veranstalter}{Super Python Talks for Life Science}{Ich organisierte eine zweiwöchige Seminarreihe für die Vermittlung von Datenanalysen mittlerer Ebene und Python{-}Programmier{-}Tutorials, die von 10 Doktoranden und Pos{-}Docs, einschließlich mir, gehalten wurden. Neben der Rekrutierung dieser Sprecher organisierte ich den Raum und die Ausrüstung für diese Sitzungen, warb für die Veranstaltungen und leitete die Sitzungen. Diese Serie war erfolgreich; Es wurde regelmäßig von 30{-}70 Forschern besucht.}%
\NewEntry{July 2016 and July 2017}{Trainer}{Einführung in die Programmierung in Python}{Dieser 4{-}tägige Workshop ist eine intensive Version des Semester{-}Python{-}Kurses, den ich an der LMU unterrichte. In diesem Zeitraum erlangen Studierende ohne Programmierkenntnisse die Fähigkeiten, die für die Datenanalyse und in Python erforderlich sind, und erklären ihren Analyse{-}Workflow.}%
\NewEntry{Summer 2016 and Summer 2017}{Data Science Dozent}{Einführung in die Programmierung in Python}{In diesem Semesterkurs, der zwei Jahre in Folge unterrichtet wurde, unterrichtete ich Programmieringenieure Datenmanagement, wissenschaftliche Datenanalyse und Programmierkenntnisse in einer neuen Sprache (Python). Neben der Organisation und Planung des Kurses habe ich auch alle Kursmaterialien, Hausaufgaben vorbereitet und ihre Abschlussarbeiten bewertet.}%
\NewEntry{Winter 2015}{Trainer}{Einführung in Matlab}{Ich plante und lehrte Matlab, mit der Programmierung von Studenten zu beginnen.}%
\NewEntry{December 2015}{Lehrassistent}{Psychophysics}{In diesem 2{-}wöchigen Blockkurs unterstützte ich Studenten bei der Programmierung und Analyse ihrer eigenen psychopysischen Experimente in Matlab, R und Excel.}%
\SubHeading{wissenschaftliche Publikationen}%
\Description{Nicholas A. Del Grosso, Justin J. Graboski, Weiwei Chen, Eduardo Blanco Hernández, Anton Sirota. ``Virtual Reality system for freely{-}moving rodents.'' bioRxiv 161232. July 2017; doi=https://doi.org/10.1101/161232}%
\Description{Broetz D., Del Grosso, N.A., Rea M., Ramos{-}Murguialday, A., Soekadar S.R., Birbaumer, N. ``A New Hand Assessment Instrument for Severely Affected Stroke Patients.''  Journal of Neurorehabilitation. 2014; 34(3), 409{-}27.}%
\Description{Benoit, J.B., Del Grosso, N.A., Yoder, J.A., Denlinger, D.L. ``Resistance to Dehydration between Bouts of Blood Feeding in the Bed Bug, Cimex Lectularius, is Enhanced by Water Conservation, Aggregation, and Quiescence.'' American Journal of Tropical Medical Hygience. May 2007; 76(5), 987{-}93.}%
\SubHeading{Konferenzpublikationen}%
\NewEntry{September 2018}{Harvard{-}LMU Young Scientists Forum}{Testing CAVE virtual reality systems for use in animal behavior research}{}%
\NewEntry{November 2017}{Society for Neuroscience}{Generalized Rat Spontaneous Behavior in a CAVE Experimental Setup.}{}%
\NewEntry{July 2017}{PyData Barcelona}{The Neuroscience Lab; A Tour Through the Eyes of a Pythonista}{}%
\NewEntry{November 2016}{Munich Interact}{Tracking Rats Exploring a Virtual World; Do They Believe what they See?}{}%
\NewEntry{July 2016}{FENS Forum of Neuroscience}{Probing Rodent Perception of Virtual Environments with Freely{-}Moving Virtual Reality}{}%
\NewEntry{June 2015}{Synergy Munich}{ratCAVE, A Novel Virtual Reality System for Freely{-}Moving Rodents}{}%
\NewEntry{March 2015}{Interact Munich}{Demonstrating a Freely{-}Moving Virtual Reality Approach for Rodent Research}{}%
\NewEntry{Nov 2014}{Society for Neuroscience}{ratCAVE, A Novel Virtual Reality System for Freely{-}Moving Rodents.}{}%
\NewEntry{Nov. 2012}{NENA Tübingen}{Interpreting (M)EEG, A First Look at Dynamic Causal Modeling.}{Introduced a probabilistic nonlinear modeling framework for interpretation of MEG and EEG data, along with the results of a pilot study in which we applied the approach.}%
\NewEntry{Nov. 2011}{NENA Tübingen}{The Intrinsic Bias During the Blind{-}Walking Task is Not Caused by an Aberrant Intrinsic Ground{-}Slope Model.}{}%
\NewEntry{April 2010}{Visual Sciences Society}{DIY ERPs, Designing inexpensive EEG systems for performing auditroy and visual cognitive studies.}{}%
\NewEntry{March 2010}{Butler Undergraduate Research Conference}{Discrimination and processing of deviant stimuli at the auditory cortex.}{}%
\NewEntry{Sep. 2009}{European Health Psychology Society}{Discrimination of attention{-}related and motor{-}related evoked activity by hemispheric comparison over the motor cortex.}{}%
\NewEntry{May 2009}{Visual Sciences Society}{Are Local Changes in Faces Really Local?}{}%
\NewEntry{May 2008}{Visual Sciences Society}{Hemispheric specialization for face processing revealed by use of thatcherized and feature{-}distorted faces.}{}%
\SubHeading{Sonstiges}%
\begin{itemize}%
\item%
\textbf{Sprachkenntnisse}: English (Mother Tongue), German (Level B1), French (Level A1-2)%
\end{itemize}%
\begin{itemize}%
\item%
\textbf{Programmieren}: Python, Matlab, C-Sharp, GLSL, R, LabView, C, Bash/Linux, LaTeX%
\end{itemize}%
\begin{itemize}%
\item%
\textbf{Grafik}: Psychopy, Neurobs Presentation, Psychophysics Toolbox, OpenGL, Pyglet, SuperLab, RatCAVE, Blender3D, Adobe Suite (Photoshop, Illustrator, and InDesign), Google SketchUp, GIMP, Inkspace%
\end{itemize}%
\begin{itemize}%
\item%
\textbf{Statistik}: Statistical Parametric Mapping (SPM), SPSS, R, Matlab Statistics Toolbox, Fieldtrip, gTec Analyze, BrainVision Analyzer%
\end{itemize}%
\begin{itemize}%
\item%
\textbf{Labor Fähigkeiten}: Ratte Neurochirurgie, Tierisches Verhaltenstraining (Ratten und Affen), In-vivo-Elektrophysiologie (Einnadelelektroden, chronisch implantierte Elektrodenarrays, nichtinvasive Arrays von EEG-Elektroden und MEG-Sensoren), Basiselektronik, Bequem mit benutzerdefinierten Laboreinrichtungen%
\end{itemize}%
\begin{itemize}%
\item%
\textbf{EEG System Erhahrungen}: BrainProducts, g.tec, Grass Instruments, CTF%
\end{itemize}%
\SubHeading{Auszeichnungen}%
\DescMarg{Oktober 2017}{Hackathon 3. Platz Gewinner und "Most Creative Team" Award beim Burda Bootcamp Event "Gesundheit und Fitness Hackathon"}%
\DescMarg{July 2017}{Hackathon Track Gewinner beim Media Lab Bayern Event "FutureLab {-} Smart Home trifft Journalismus"}%
\DescMarg{April 2017}{Hackathon Gewinner beim Burda Bootcamp Event "Love Hackathon"}%
\DescMarg{2016}{Best Talk Award auf der Interact München Conference}%
\DescMarg{2015}{Best Poster Award auf der Interact Munich Conference}%
\DescMarg{2011}{National Science Foundation Graduate Research Graduiertenstipendium}%
\DescMarg{2008}{NSF Neuroscience REU Fellowship at Duke Graduiertenstipendium}%
\end{cv}%
\end{document}